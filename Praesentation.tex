\documentclass{beamer}
\usetheme{Dresden}
\usepackage[ngerman]{babel}
\usepackage{latexsym}
\usepackage{amsmath,amssymb,amsthm}
\usepackage{xcolor}
\usepackage{tikz}
\usepackage{pgfplots}
\usepackage{mhchem}
\usepackage{braket}
\usepackage{hyperref}
\usepackage{gensymb}

% \usepackage{beamerthemesplit} // Activate for custom appearance

\title{Asymmetrien der Physik}
\author{Matteo Schmider}
\date{\today}

\begin{document}

\frame{\titlepage}

\begin{frame}
\frametitle{Inhaltsverzeichnis}\tableofcontents
\end{frame}

\section{Einführung}
\subsection{Symmetrische Eigenschaften der Physik}
\frame
{
  \frametitle{Features of the Beamer Class}

}
\subsection{Symmetrieeigenwerte}
\frame
{
  \frametitle{Features of the Beamer Class}

}
\section{CPT-Theorem}
\subsection{C-Symmetrie}
\frame
{
  \frametitle{Features of the Beamer Class}

}
\subsection{P-Symmetrie}
\frame
{
  \frametitle{Features of the Beamer Class}

}
\subsection{T-Symmetrie}
\frame
{
  \frametitle{Allgemein}
  \begin{itemize}
    \item Invarianz unter Zeitumkehr, d.h. unter der Transformation
      \begin{align}
        T: t \to -t
      \end{align}
      \pause
    \item Aufgrund der Erscheinung in realen Beispielen Bewegungsumkehr genannt
    \item Vorstellbar als das rückwärts Abspielen eines Videos
  \end{itemize}
}
\frame
{
  \frametitle{Beispiel Fadenpendel}
  \begin{columns}
    \begin{column}{0.5\textwidth}
      Auslenkung zur Zeit $t_{0}$:\\
      Geschwindigkeit zur Zeit $t_{0}$:
    \end{column}
    \begin{column}{0.5\textwidth}  %%<--- here
      10\degree\\
      0
    \end{column}
  \end{columns}
}
\subsection{Logik der Zusammensetzung}
\frame
{
  \frametitle{CPT als \glqq{höhere}\grqq{} Symmetrie}
  \begin{itemize}
    \item Idee:\\Invarianz unter Hintereinanderausführung der Transformationen \pause
    \item Invarianz-Regel:\\Verletzung einer Symmetrie wird durch die zwei weiteren Transformationen aufgehoben 
  \end{itemize}
}
\section{Symmetriebrechungen}
\subsection{Paritätsverletzung}
\frame
{
  \begin{figure} [!ht]
  \begin{tikzpicture}
  \begin{axis}[
    view={35}{15},
    axis lines=center,
    width=10cm,height=9cm,
    xmin=-2,xmax=2,ymin=-2,ymax=2,zmin=-3,zmax=2
  ]
  % plot dots for the two points
  \addplot3 [only marks] coordinates {(1,1,1) (-1,-1,-1)};
  % label points
  \node [above right] at (axis cs:1,1,1) {$\ce{^{60}_{27}Co}$};
  \node [below left] at (axis cs:-1,-1,-1) {$\ce{^{60}_{27}Co}'$};
  \draw [->, thick] (axis cs:1,1,1) -- (axis cs:1,1,2);
  \draw [->, thick] (axis cs:-1,-1,-1) -- (axis cs:-1,-1,0);
  \node [right] at (axis cs:1,1,2) {Spin};
  \node [below left] at (axis cs:-1,-1,0) {$Spin'$};
  \draw [->, dotted] (axis cs:1,1,1) -- (axis cs:1,1,0);
  \draw [->, dotted] (axis cs:-1,-1,-1) -- (axis cs:-1,-1,-2);
  \node [below right] at (axis cs:1,1,0) {$e_1^-$};
  \node [below right] at (axis cs:-1,-1,-2) {$e_2^-$};
  \end{axis}
  \end{tikzpicture}
  \caption{P-Verletzung}
  \end{figure}
}
\frame
{
  \frametitle{Folgerungen}
  \begin{itemize}
    \item Erwartung: Gleiche Menge an Elektronen auf beiden Seiten \pause
    \item Realität: Es verlassen mehr Elektronen die Atome entgegen der Spin-Richtung \pause
  \end{itemize}
  $\Rightarrow$ Parität verletzt
}
\subsection{CP-Verletzung}
\frame
{
  \frametitle{Features of the Beamer Class}

}
\subsection{T-Verletzung}
\frame
{
  \frametitle{Features of the Beamer Class}

}
\end{document}
