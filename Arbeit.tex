\documentclass[12pt,a4paper]{scrartcl}
% scrartcl ist eine abgeleitete Artikel-Klasse im Koma-Skript
% zur Kontrolle des Umbruchs Klassenoption draft verwenden


% die folgenden Packete erlauben den Gebrauch von Umlauten und ß
% in der Latex Datei
\usepackage[utf8]{inputenc}
\usepackage[T1]{fontenc}
\usepackage[ngerman]{babel}


\usepackage[pdftex]{graphicx}
\usepackage{latexsym}
\usepackage{amsmath,amssymb,amsthm}
\usepackage{xcolor}


% Abstand obere Blattkante zur Kopfzeile ist 2.54cm - 15mm
\setlength{\topmargin}{-15mm}


% Umgebungen für Definitionen, Sätze, usw.
% Es werden Sätze, Definitionen etc innerhalb einer Section mit
% 1.1, 1.2 etc durchnummeriert, ebenso die Gleichungen mit (1.1), (1.2) ..
\newtheorem{Satz}{Satz}[section]
\newtheorem{Definition}[Satz]{Definition}
\newtheorem{Lemma}[Satz]{Lemma}

\numberwithin{equation}{section}

% einige Abkuerzungen
\newcommand{\C}{\mathbb{C}} % komplexe
\newcommand{\K}{\mathbb{K}} % komplexe
\newcommand{\R}{\mathbb{R}} % reelle
\newcommand{\Q}{\mathbb{Q}} % rationale
\newcommand{\Z}{\mathbb{Z}} % ganze
\newcommand{\N}{\mathbb{N}} % natuerliche



\begin{document}
  % Keine Seitenzahlen im Vorspann
  \pagestyle{empty}

  % Titelblatt der Arbeit
  \begin{titlepage}

    \vspace*{2cm}

 \begin{center} \large

    Seminararbeit
    \vspace*{2cm}

    {\huge Zeitreisen im Spiegel}
    \vspace*{2.5cm}

    Matteo Schmider
    \vspace*{1.5cm}

    05.11.2019
    \vspace*{4.5cm}


    Betreuung: Johannes Klees \\[1cm]
    W-Seminar Wissenschaft und Comedy \\[1cm]
		Simpert-Kraemer Gymnasium Krumbach
  \end{center}
\end{titlepage}



  % Inhaltsverzeichnis
  \tableofcontents

\newpage



  % Ab sofort Seitenzahlen in der Kopfzeile anzeigen
  \pagestyle{headings}

  \section{Einleitung: Comedy und Wissenschaft}

Die Sache mit der Zeit...
Erst vor kurzem strahlte der deutsche Fernsehsender ProSieben die latzte Staffel der Serie "The Big Bang Theory" aus. Bekanntermaßen handelt die Serie von der WG der beiden fiktiven Physiker Dr. Leonard Leakey Hofstadter und Dr. Dr. Sheldon Lee Cooper und ihrem "Nerd-Dasein", was im starken Widerspruch zum Charakter ihrer Nachbarin, Penny steht. Die Serie ist unter anderem für die vielen, oft komödischen, Anspielungen auf reale Physikalische Phänomene und Theorien. \iffalse 1-5 Zeilen Nacherzählung der Serie \fi Ihren Höhepunkt erreicht die Serie jedoch in ihrer letzten Folge, deren Hauptinhalt die Nobelpreisverleihung an Sheldon und Amy für ihre Theorie der Super-Asymmetrie. In dieser Arbeit möchte ich einen Einblick in die Bedeutung von Symmetrien und Asymmetrien in der Physik geben, anhand des erstaunlichen Phänomens der T-Verletzung. Die T-Verletzung ist eine Schlussfolgerung aus der experimentell nachgewiesenen CP-Verletzung und der Gültigkeit der CPT-Theorie, wobei die CP-Verletzung und damit auch die T-Verletzung historisch als unmöglich betrachtet wurden. \iffalse Letzte zwei Sätze zu abstrakt, müssen besser zu lockererem Stil der Einleitung passen, Einleitung soll trotzdem noch möglichst genau das Thema vorbereiten \fi

%%%%%%%%%%%%%%%%%%%%%%%%%%%%%%%%%
  \newpage  % neuer Abschnitt auf neue Seite, kann auch entfallen
%%%%%%%%%%%%%%%%%%%%%%%%%%%%%%%%%

  \section{Allgemeine Transformation einer physikalischen Größe}

Eine Transformation f einer physikalischen Größe x ist im Allgemeinen immer eine Abbildung der Art
\begin{align}
    f: x' \to x
\end{align}
Um beispielsweise die Masse eines Objekts zu verdoppeln:
\begin{align}
    f(m) = 2m
\end{align}
Ein Beispiel für eine wichtige Transformation in der Physik ist die Lorentz-Transformation:
\begin{align}
  \begin{split}
    t' &= \gamma(t - \frac{v_x}{c^2}x) \\ x' &= \gamma(x -v_x t) \\ y' &= y \\ z' &= z \\ v_x' &= -v_x
  \end{split}
\end{align}
Diese Transformation beschreibt, wie ein Inertialsystem A am Ort $ \vec{r} = \begin{pmatrix}x'\\y'\\z'\end{pmatrix} $, mit konstanter Geschwindigkeit $v_x'$ in x-Richtung relativ zum Beobachter-System B, zur Zeit t' und B ineinander überführt werden können. Die Lorentz-Transformation ist für die relativistische Raumzeit gültig und ersetzt darin die Galilei-Transformation.
Wichtig für den weiteren Verlauf der Arbeit ist jedoch vor allem die grundlegende Transformation der Spiegelung:
\begin{align}
    f: x' \to - x
\end{align}

%%%%%%%%%%%%%%%%%%%%%%%%%%%%%%%%%
  \newpage  % neuer Abschnitt auf neue Seite, kann auch entfallen
%%%%%%%%%%%%%%%%%%%%%%%%%%%%%%%%%

  \section{C-Transformation und C-Invarianz}

Die C-Transformation hat ihren Namen von "charge" (engl. Ladung) und spiegelt die Ladung eines betrachteten Teilchens. Nachdem Elementarteilchen eine festgelegte Ladung besitzen, ist diese Operation in der Realität als das Ersetzen des Teilchens durch das dazugehörige Antiteilchen zu verstehen. Das Antiteilchen besitzt immer dieselbe Masse und Eigenschaften des Ausgangs-Teilchens,

%%%%%%%%%%%%%%%%%%%%%%%%%%%%%%%%%
  \newpage  % neuer Abschnitt auf neue Seite, kann auch entfallen
%%%%%%%%%%%%%%%%%%%%%%%%%%%%%%%%%

  \section{P-Transformation und P-Invarianz}

%%%%%%%%%%%%%%%%%%%%%%%%%%%%%%%%%
  \newpage  % neuer Abschnitt auf neue Seite, kann auch entfallen
%%%%%%%%%%%%%%%%%%%%%%%%%%%%%%%%%

  \section{T-Transformation und T-Invarianz}

%%%%%%%%%%%%%%%%%%%%%%%%%%%%%%%%%
  \newpage  % neuer Abschnitt auf neue Seite, kann auch entfallen
%%%%%%%%%%%%%%%%%%%%%%%%%%%%%%%%%

  \section{CP-Transformation als Kombination zweier Transformationen}

%%%%%%%%%%%%%%%%%%%%%%%%%%%%%%%%%
  \newpage  % neuer Abschnitt auf neue Seite, kann auch entfallen
%%%%%%%%%%%%%%%%%%%%%%%%%%%%%%%%%

  \section{experimenteller Nachweis der CP-Verletzung}

%%%%%%%%%%%%%%%%%%%%%%%%%%%%%%%%%
  \newpage  % neuer Abschnitt auf neue Seite, kann auch entfallen
%%%%%%%%%%%%%%%%%%%%%%%%%%%%%%%%%

  \section{T-Verletzung als Folge der CP-Verletzung und damit erwiesene Zeitrichtung}

%%%%%%%%%%%%%%%%%%%%%%%%%%%%%%%%%
  \newpage  % neuer Abschnitt auf neue Seite, kann auch entfallen
%%%%%%%%%%%%%%%%%%%%%%%%%%%%%%%%%

  \section{Späterer Nachweis der T-Verletzung (in kurzer Form)}

  % Literaturverzeichnis (beginnt auf einer ungeraden Seite)
  \newpage

\begin{thebibliography}{Lam00}

\end{thebibliography}


  % ggf. hier Tabelle mit Symbolen
  % (kann auch auf das Inhaltsverzeichnis folgen)

\newpage

 \thispagestyle{empty}


\vspace*{8cm}


\section*{Erklärung}

Ich  versichere  wahrheitsgemäß,  die  Arbeit selbstständig verfasst,  alle  benutzten  Hilfsmittel  vollständig  und  genau  angegeben  und  alles kenntlich  gemacht  zu  haben,  was  aus  Arbeiten  anderer  unverändert  oder  mit  Abänderungen entnommen  wurde,  sowie die Satzung  des  KIT  zur  Sicherung guter wissenschaftlicher Praxis in der jeweils gültigen Fassung beachtet zu haben.
\\[2ex]

\noindent
Ort, den Datum\\[5ex]

% Unterschrift (handgeschrieben)



\end{document}
